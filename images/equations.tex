\documentclass{article}
\usepackage{mathtools}
\usepackage{amssymb}
\DeclareMathOperator*{\argmax}{arg\,max}
\DeclareMathOperator*{\argmin}{arg\,min}
\begin{document}

% Equation 1
\begin{equation}
	V(\boldsymbol{u}) = -k_{\text{B}} T \ln \frac{ p(\boldsymbol{u}) }
		{ \max\limits_{\boldsymbol{u}\in\mathbb{V}} p(\boldsymbol{u}) }
\end{equation}

% Equation 2
\begin{eqnarray}
	& V(\boldsymbol{u}) = -k_{\text{B}} T \ln \frac{ \varrho(\boldsymbol{u}) }{ \varrho_0 } \\[1.25ex]
	& \varrho_0 = \min\limits_{\boldsymbol{u}\in\mathbb{V}} \varrho(\boldsymbol{u})  \quad\lor\quad
		\varrho_0 = \max\limits_{\boldsymbol{u}\in\mathbb{V}} \varrho(\boldsymbol{u})  \quad\lor\quad
		\varrho_0 = \varrho_{\text{loc}} \nonumber
\end{eqnarray}

% Equation 3
\begin{equation}
	\Delta_{\pm}V(\boldsymbol{u}) = -k_{\text{B}} T \ln
		\frac{ 1\mp \left| \frac{ \Delta p(\boldsymbol{u}) }{ p(\boldsymbol{u}) }\right| }{ 1 \pm \left|
		\frac{ \Delta p \left( \argmax\limits_{\boldsymbol{u}\in\mathbb{V}} p(\boldsymbol{u}) \right) }
		{p \left( \argmax\limits_{\boldsymbol{u}\in\mathbb{V}} p(\boldsymbol{u}) \right) } \right| }
\end{equation}

\end{document}